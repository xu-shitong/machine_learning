\documentclass[UTF8]{ctexart}
\usepackage{geometry}
\usepackage{amsmath}
\usepackage{graphicx} %插入图片的宏包
\usepackage{float} %设置图片浮动位置的宏包
\geometry{a4paper,scale=0.8}
\sectionfont{\bfseries\Large\raggedright}

\title{pandas笔记}
\author{徐世桐}
\date{}
\begin{document}
\maketitle

% ----------------------------------------------------------------------
% |                              import                                |
% ----------------------------------------------------------------------
\section{import}
\noindent \texttt{import pandas}\\
\texttt{import pandas}\\
\texttt{import matplotlib.pyplot as plt}\\
\texttt{from sklearn.datasets import fetch\_openml}\\
\texttt{from pandas.plotting import scatter\_matrix}
% ----------------------------------------------------------------------
% |                            使用csv数据                              |
% ----------------------------------------------------------------------
\section{使用csv数据}
\noindent \texttt{data = pandas.read\_csv('.csv文件路径')}

  $data$为$pandas.Datafram$类型\\
\texttt{data = pandas.DataFrame(data={'特征名': python数组, ...})}

   根据python数组创建datafram\\
\texttt{data.head()} // 显示前5组数据\\
\texttt{data.info()} // 显示每一特征的信息\ 数据类型\\
\texttt{data['特征名']} 
  
  显示某一特征的所有数据,输出$pandas.Series$\\
\texttt{data['特征名'].value\_counts()} // 显示此特征所有取值,对每一取值显示对应样本数,输出$pandas.Series$\\
\texttt{data.describe()} // 显示每一特征统计信息,输出$pandas.Datafram$\\
\texttt{data.hist(BINS, FIGSIZE)}\\
\texttt{matplotlib.pyplot.show()}

  将\texttt{data}中每一特征统计结果用直方图表示

  BINS: \texttt{bins=N} 直方图将被分为$N$个值点,有$N-1$个区间

  FIGSIZE: \texttt{figsize=(宽, 高)} 定义每一特征的直方图形状\\
\texttt{data.iloc[index\_array]}

  对$index_array$每一$index$得到$data$中对应位置的样本信息,输出$pandas.Series$\\
\texttt{data.loc[row\_array]}

  类似$iloc$,但根据行标签进行取样,非行$index$号\\
\texttt{data\_copy = data.copy()} // 复制数据\\
\texttt{series.sort\_values(ASCENDING)}

  对一个$pandas.Series$输出排序后的数据,输出$pandas.Series$

  ASCENDING: 取boolean值,是否按递增顺序输出\\
\texttt{data.corr()} 
  
  对所有特征两两求correlation

  输出$pandas.Datafram$,通过\texttt{data.corr()['特征名']}得到一个特征关于其他特征的corr值

% ----------------------------------------------------------------------
% |                              csv绘图                                |
% ----------------------------------------------------------------------
\section{csv绘图}
\noindent \texttt{data.plot(KIND, X, Y, ALPHA*, S*, C*, CMAP*, FIGSIZE*)}

  调用后使用\texttt{plt.show()}显示图像

  KIND: 定义图表类型
  
  \quad \texttt{kind='scatter'} 描点图

  X: \texttt{x='特征名'},Y: \texttt{y='特征名'}

  \quad 定义横纵坐标采用哪一特征下的值

  ALPHA: \texttt{alpha=0.1} 点填充设为半透明,使点浓度高处颜色深

  S: \texttt{s=data['特征名']} 用点大小表示特征值高低
  
  C: \texttt{c='特征名'} 用点颜色表示特征值高低,和CMP同时使用

  CMAP: \texttt{cmp=plt.get\_cmap('jet')} 使用plt内定义的jet色谱。通过点颜色表示C中选择的特征值值高低

  FIGSIZE: \texttt{figsize=(宽, 高)}\\
\texttt{scatter\_matrix(DATA, figsize=(宽, 高))}

  同时显示多组散点图

  DATA: 为$pandas.Datafram$
  
  \quad \texttt{=data} 对所有\texttt{data}中特征两两画图

  \quad \texttt{=data['特征1', '特征2', ...]} 选择某些特征两两画图

% ----------------------------------------------------------------------
% |                              数据操作                               |
% ----------------------------------------------------------------------
\section{数据操作}
\noindent \texttt{np.c\_[a, b, ...]}

  创建numpy.ndarray类型,\textbf{和python array不同}

  \texttt{a, b, ...} 类型相同,shape相同\\
\texttt{pandas.Datafram}可调\texttt{.shape} \texttt{.values}转\texttt{np.ndarray}\\
\texttt{pandas.Series}可调\texttt{.shape} \texttt{.values}转\texttt{np.ndarray}\\
\texttt{np.ndarray}可调\texttt{.reshape} \texttt{.shape}

% ----------------------------------------------------------------------
% |                            MNIST数据集                              |
% ----------------------------------------------------------------------
\section{MNIST数据集}
\noindent \texttt{mnist = fetch\_openml('mnist\_784', version=1)} //得到手写字母的训练集\\
\texttt{X = mnist['data'], y = mnist['target']} // 得到$pandas.Datafram$特征集,label集\\
\texttt{plt.imshow(X.iloc[0].values.reshape(28, 28))} // 绘制第一个图像,使用\texttt{plt.show()}显示\\


\end{document}
