\documentclass[UTF8]{ctexart}
\usepackage{geometry}
\usepackage{amsmath}
\usepackage{graphicx} %插入图片的宏包
\usepackage{float} %设置图片浮动位置的宏包
\usepackage{setspace}
\geometry{a4paper,scale=0.8}
\sectionfont{\bfseries\Large\raggedright}

\title{torch笔记}
\author{徐世桐}
\date{}
\begin{document}
\maketitle

% ----------------------------------------------------------------------
% |                              import                                |
% ----------------------------------------------------------------------
\section{import}
\noindent \texttt{from torch import nn}\\
\texttt{import torch.nn.functional as F}\\
\texttt{import torch.optim as optim}


% ----------------------------------------------------------------------
% |                           tensor数据类型                             |
% ----------------------------------------------------------------------
\section{tensor数据类型}
\noindent \texttt{torch.arange}

  \texttt{torch.arange(a)} 得到Tensor$[0, 1, ..., \lfloor a \rfloor]$

  \texttt{torch.arange(a, b)} 得到Tensor$[a, a+1, ..., a+n]$,n为整数且$a+n < b$

  \texttt{torch.arange(a, b, c)} 得到Tensor$[a, a+c, ..., a+nc]$,n为整数且$a+nc < b$\\
\texttt{torch.from\_numpy(NDArray)} 从\texttt{NDArray}创建Tensor\\
\texttt{torch.mm(Tensor, Tensor)} tensor矩阵乘法\\
\texttt{+-*/} 同NDArray使用广播机制\\
\texttt{Tensor.reshape()} 改变形状,\textbf{新形状元素数必须等于输入元素数}\\
\texttt{torch.random(MEAN, STD, SIZE*)}

  \texttt{size=($x_1, x_2, ...$)} 限定输出张量形状

  \texttt{mean=Tensor}, \texttt{std=Tensor/const} 当没有限定size时\texttt{mean}必为float Tensor,形状和输出形状相同。

  \texttt{mean=Tensor/const}, \texttt{std=Tensor/const} 当限定size后\texttt{mean, std}可为const或单个值的Tensor\\
\texttt{torch.rand(SIZE*)}

  得到SIZE形状的随机数张量,每一元素$\in [0,1)$。SIZE无定义则得到const随机数

  代替torch.uniform功能\\
\texttt{dataset = torch.utils.data.TensorDataset(样本Tensor, 标签Tensor)}\\
\texttt{dataiter = torch.utils.data.DataLoader(dataset, batch\_size=批量大小, shuffle=True)}

  使用torch进行批量迭代\\


% ----------------------------------------------------------------------
% |                           torch神经网络                              |
% ----------------------------------------------------------------------
\section{torch神经网络}
\noindent \texttt{net = nn.Sequential()}\\
神经网络定义:\texttt{net.add\_module('层名', 层)}\\
层定义:

  \texttt{nn.Linear(输入节点数,输出节点数)} 定义全连接层

  \texttt{}\\
\textbf{可使用层直接进行前向计算,训练函数中使用[layer.weight, net.bias]传入参数}\\
\texttt{net.weight/bias.data.fill\_(值)} 对层中所有权重/偏差赋值\\
\texttt{nn.init.xavier\_uniform(net.weight/.bias)} 对层中所有权重/偏差使用xavier初始化\\
\texttt{loss = nn.MSELoss()} 平方代价函数\\
\texttt{trainer = optim.SGD(net.parameters(), lr=学习率)} SGD迭代函数

  trainer.step() 进行迭代\\
\texttt{net.parameters()} 得到权重

  

  


\end{document}
