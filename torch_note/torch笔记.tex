\documentclass[UTF8]{ctexart}
\usepackage{geometry}
\usepackage{amsmath}
\usepackage{graphicx} %插入图片的宏包
\usepackage{float} %设置图片浮动位置的宏包
\geometry{a4paper,scale=0.8}
\sectionfont{\bfseries\Large\raggedright}

\title{torch笔记}
\author{徐世桐}
\date{}
\begin{document}
\maketitle

% ----------------------------------------------------------------------
% |                              import                                |
% ----------------------------------------------------------------------
\section{import}



% ----------------------------------------------------------------------
% |                           tensor数据类型                             |
% ----------------------------------------------------------------------
\section{tensor数据类型}
\noindent \texttt{torch.arange}

  \texttt{torch.arange(a)} 得到Tensor$[0, 1, ..., \lfloor a \rfloor]$

  \texttt{torch.arange(a, b)} 得到Tensor$[a, a+1, ..., a+n]$,n为整数且$a+n < b$

  \texttt{torch.arange(a, b, c)} 得到Tensor$[a, a+c, ..., a+nc]$,n为整数且$a+nc < b$\\
\texttt{torch.from\_numpy(NDArray)} 从\texttt{NDArray}创建Tensor\\
\texttt{torch.mm(Tensor, Tensor)} tensor矩阵乘法\\
\texttt{+-*/} 同NDArray使用广播机制\\
\texttt{Tensor.reshape()} 改变形状,\textbf{新形状元素数必须等于输入元素数}\\
\texttt{torch.random(MEAN, STD, SIZE*)}

  \texttt{size=($x_1, x_2, ...$)} 限定输出张量形状

  \texttt{mean=Tensor}, \texttt{std=Tensor/const} 当没有限定size时\texttt{mean}必为float Tensor,形状和输出形状相同。

  \texttt{mean=Tensor/const}, \texttt{std=Tensor/const} 当限定size后\texttt{mean, std}可为const或单个值的Tensor\\
\texttt{torch.rand(SIZE*)}

  得到SIZE形状的随机数张量,每一元素$\in [0,1)$。SIZE无定义则得到const随机数

  代替torch.uniform功能\\
\textbf{}


\end{document}
